%----------------------------------------------------------------------------------------
%   Доорх хэсгийг өөрчлөх шаардлагагүй
%----------------------------------------------------------------------------------------
%!TEX TS-program = xelatex
%!TEX encoding = UTF-8 Unicode
\documentclass[12pt,A4]{report}

\usepackage{fontspec,xltxtra,xunicode}
\setmainfont[Ligatures=TeX]{Times New Roman}
\setsansfont{Arial}

% \usepackage[utf8x]{inputenc}
% \usepackage[mongolian]{babel}
%\usepackage{natbib}
\usepackage{geometry}
%\usepackage{fancyheadings} fancyheadings is obsolete: replaced by fancyhdr. JL
\usepackage{fancyhdr}
\usepackage{float}
\usepackage{afterpage}
\usepackage{graphicx}
\usepackage{amsmath,amssymb,amsbsy}
\usepackage{dcolumn,array}
\usepackage{tocloft}
\usepackage{dics}
\usepackage{nomencl}
\usepackage{upgreek}
\newcommand{\argmin}{\arg\!\min}
\usepackage{mathtools}
\usepackage[hidelinks]{hyperref}

\usepackage{algorithm}
\usepackage{algpseudocode}

\usepackage{listings}
\DeclarePairedDelimiter\abs{\lvert}{\rvert}%
\makeatletter
\usepackage{caption}
\captionsetup[table]{belowskip=0.5pt}
\usepackage{subfiles}
\usepackage{subcaption} 

\usepackage{listings}
\renewcommand{\lstlistingname}{Код}
\renewcommand{\lstlistlistingname}{\lstlistingname ын жагсаалт}

\usepackage{color}
\definecolor{codegreen}{rgb}{0,0.6,0}
\definecolor{codegray}{rgb}{0.5,0.5,0.5}
\definecolor{codepurple}{rgb}{0.58,0,0.82}
\definecolor{backcolour}{rgb}{0.99,0.99,0.99}
 
\lstdefinestyle{mystyle}{
    basicstyle=\ttfamily\small,
    backgroundcolor=\color{backcolour},   
    commentstyle=\color{codegreen},
    keywordstyle=\color{magenta},
    numberstyle=\tiny\color{codegray},
    stringstyle=\color{codepurple},
    %basicstyle=\footnotesize,
    breakatwhitespace=false,         
    breaklines=true,                 
    captionpos=b,                    
    keepspaces=false,                 
    numbers=left,                    
    numbersep=10pt,                  
    showspaces=false,                
    showstringspaces=true,
    showtabs=false,                  
    tabsize=2
}
 
\lstset{style=mystyle, label=DescriptiveLabel} 

\let\oldabs\abs
\def\abs{\@ifstar{\oldabs}{\oldabs*}}
\makenomenclature
\begin{document}


%----------------------------------------------------------------------------------------
%   Өөрийн мэдээллээ оруулах хэсэг
%----------------------------------------------------------------------------------------

% Дадлагын ажлын сэдэв
\title{БД СИСТЕМС ЭЙЖИА ХХК-Д ХИЙСЭН ДАДЛАГЫН АЖЛЫН ТАЙЛАН}
% Дадлагын ажлын англи нэр
\titleEng{Internship Report at BD Systems Asia LLC}
% Өөрийн овог нэрийг бүтнээр нь бичнэ
\author{[Таны овог нэр]}
% Өөрийн овгийн эхний үсэг нэрээ бичнэ
\authorShort{[Т.Нэр]}
% Удирдагчийн зэрэг цол овгийн эхний үсэг нэр
\supervisor{[Багш]}
% Хамтарсан удирдагчийн зэрэг цол овгийн эхний үсэг нэр
%\cosupervisor{Др. Г.Амарсанаа}

% СиСи дугаар 
\sisiId{[Оюутны дугаар]}
% Их сургуулийн нэр
\university{МОНГОЛ УЛСЫН ИХ СУРГУУЛЬ}
% Бүрэлдэхүүн сургуулийн нэр
\faculty{МЭДЭЭЛЭЛ, КОМПЬЮТЕРЫН УХААНЫ ТЭНХИМ}
% Тэнхимийн нэр
\department{МЭДЭЭЛЭЛ, КОМПЬЮТЕРЫН УХААНЫ ТЭНХИМ}
% Зэргийн нэр
\degreeName{Дадлагын ажлын тайлан}
% Суралцаж буй хөтөлбөрийн нэр
\programeName{Компьютерын ухаан}
% Хэвлэгдсэн газар
\cityName{Улаанбаатар}
% Хэвлэгдсэн огноо
\gradyear{2025 оны 09 сар}


%----------------------------------------------------------------------------------------
%   Доорх хэсгийг өөрчлөх шаардлагагүй
%----------------------------------------------------------------------------------------
%----------------------------------------------------------------------------------------
%   Title page setup
%----------------------------------------------------------------------------------------

\newgeometry{top=3cm,bottom=3cm,left=3cm,right=3cm}

\begin{titlepage}
\begin{center}

\vspace*{1cm}

{\Large \university}

\vspace{0.5cm}

{\large \faculty}

\vspace{0.3cm}

{\normalsize \department}

\vspace{3cm}

{\huge \@title}

\vspace{0.5cm}

{\Large \titleEng}

\vspace{3cm}

{\large \degreeName}

\vspace{0.5cm}

{\normalsize \programeName}

\vspace{2cm}

\begin{tabular}{ll}
Гүйцэтгэсэн: & \authorShort \\
             & \sisiId \\[1cm]
Удирдагч:    & \supervisor \\
\ifcsname cosupervisor\endcsname
Хамтарсан удирдагч: & \cosupervisor \\
\fi
\end{tabular}

\vfill

{\large \cityName}

\vspace{0.5cm}

{\large \gradyear}

\end{center}
\end{titlepage}

\restoregeometry

%----------------------------------------------------------------------------------------
%   Setup page numbering and headers
%----------------------------------------------------------------------------------------

% Агуулгын хүснэгт, зурган жагсаалт зэрэгт римийн тоо ашиглах
\pagenumbering{roman}
\setcounter{page}{1}

% Header тохиргоо
\pagestyle{fancy}
\fancyhf{}
\fancyhead[C]{\small БД Системс Эйжиа ХХК Дадлагын Ажлын Тайлан}
\fancyfoot[C]{\thepage}
\renewcommand{\headrulewidth}{0.4pt}

% Chapter heading тохиргоо
\renewcommand{\chaptername}{БҮЛЭГ}

% Агуулгын хүснэгт
\renewcommand{\contentsname}{АГУУЛГА}
\renewcommand{\listfigurename}{ЗУРГИЙН ЖАГСААЛТ}
\renewcommand{\listtablename}{ХҮСНЭГТИЙН ЖАГСААЛТ}

% Зураг, хүснэгтийн нэрийг монголоор
\renewcommand{\figurename}{Зураг}
\renewcommand{\tablename}{Хүснэгт}

% Bibliography нэрийг монголоор
\renewcommand{\bibname}{НОМ ЗҮЙ} 

% Удиртгалыг оруулж ирэх ба abstract.tex файлд удиртгалаа бичнэ
\documentclass[main.tex]{subfiles}

\begin{document}

\chapter*{УДИРТГАЛ}
\addcontentsline{toc}{chapter}{УДИРТГАЛ}

\section*{Дадлагын үндэслэл ба судлах шаардлага}

Орчин үеийн дижитал ертөнцөд мэдээллийн аюулгүй байдал нь аливаа байгууллага, хувь хүний хувьд маш чухал асуудал болж байна. Өдөр тутмын амьдралд интернэт, вэб аппликейшн зэрэг технологийн ашиглалт өргөжин тархаж буй энэ үед кибер аюул заналхийлэл улам бүр нэмэгдэж байна.

Дэлхийн статистикийн дагуу жилд 4.7 тэрбум хүн интернэт ашигладаг бол 1.8 тэрбум вэб аппликейшн ажиллаж байна \cite{2}. Энэ хурдацтай өсөлт нь аюулгүй байдлын шинэ сорилтуудыг бий болгож байна. OWASP (Open Web Application Security Project) байгууллагын судалгаагаар дундаж вэб аппликейшнд 33 аюулгүй байдлын сул тал байдаг бөгөөд эдгээрийн 19\% нь системийг бүрэн хяналтад авах боломжтой \cite{3}.

БД Системс Эйжиа ХХК нь Bitdefender-ийн Монгол дахь албан ёсны дистрибьютор бөгөөд мэдээллийн аюулгүй байдлын шийдлүүдийг санал болгодог. Энэ дадлагын ажлын үеэр би WebSecPen нэртэй автомат вэб аппликейшний аюулгүй байдлын мониторинг, шинжилгээний систем хөгжүүлэх боломж олдсон юм.

\section*{Дадлагын зорилго ба зорилтууд}

\textbf{Үндсэн зорилго:} WebSecPen автомат вэб аппликейшний аюулгүй байдлын шинжилгээ, мониторингийн систем хөгжүүлэн мэдээллийн аюулгүй байдлын салбарт практик туршлага олж авах.

\textbf{Тодорхой зорилтууд:}

\begin{enumerate}
    \item \textbf{Техник зорилтууд:}
    \begin{itemize}
        \item Flask backend ашиглан REST API хөгжүүлэх
        \item React frontend дээр орчин үеийн хэрэглэгчийн интерфейс бүтээх
        \item OWASP ZAP сканнер интеграци хийх
        \item HuggingFace AI моделийг ашиглан зөвлөмж системийн бүтээх
        \item Аюулгүй байдлын тайлан автомат үүсгэх системийг хэрэгжүүлэх
    \end{itemize}
    
    \item \textbf{Боловсролын зорилтууд:}
    \begin{itemize}
        \item Вэб аппликейшний аюулгүй байдлын талаар гүнзгий мэдлэг олж авах
        \item Орчин үеийн веб технологиудтай ажиллах чадвар эзэмших
        \item Автомат тестлэлт, CI/CD pipeline ашиглах арга барил сурах
        \item Мэдээллийн аюулгүй байдлын стандарт, практиктай танилцах
    \end{itemize}
    
    \item \textbf{Мэргэжлийн зорилтууд:}
    \begin{itemize}
        \item Компанийн бодит орчинд ажиллах туршлага олж авах
        \item Багаар ажиллах, төсөл удирдах чадвар хөгжүүлэх
        \item Клиенттэй харилцах, хэрэгцээ шинжлэх чадвар эзэмших
    \end{itemize}
\end{enumerate}

\section*{Дадлагын хамрах хүрээ}

\textbf{Техник хамрах хүрээ:}

\begin{itemize}
    \item \textbf{Backend хөгжүүлэлт:} Python Flask framework, SQLAlchemy ORM, JWT токен аутентификаци, RESTful API загвар, Celery асинхрон ажлын дараалал
    \item \textbf{Frontend хөгжүүлэлт:} React library (v18+), Material-UI дизайн систем, Axios HTTP клиент, React Router навигаци, Chart.js график харуулалт
    \item \textbf{Аюулгүй байдлын багажууд:} OWASP ZAP автомат сканнер, HuggingFace NLP модел, Redis кэш сервис, PostgreSQL өгөгдлийн сан
    \item \textbf{DevOps ба deployment:} Docker контейнержилт, GitHub Actions CI/CD, Render платформд deployment, Nginx вэб сервер
\end{itemize}

\textbf{Функционал хамрах хүрээ:}

\begin{itemize}
    \item Вэб сайтын аюулгүй байдлын автомат шалгалт
    \item Илрүүлсэн сул талуудын дэлгэрэнгүй тайлан
    \item AI ашиглан зөвлөмж, заавар өгөх систем
    \item Хэрэглэгчийн эрхийн удирдлага
    \item Скан түүхийн хадгалалт, хайлт
    \item PDF тайлан үүсгэх функц
    \item Админ хяналтын самбар
\end{itemize}

\section*{Тайланы бүтэц}

Энэхүү тайлан дараах бүтэцтэй байна:

\textbf{1-р бүлэг} - БД Системс Эйжиа ХХК компанийн танилцуулга, Bitdefender технологи, компанийн зорилго үйл ажиллагааг тайлбарлана.

\textbf{2-р бүлэг} - Дадлагын хөтөлбөр, хугацаа, хариуцлагатай ажилтнууд, ашигласан арга зүйг тодорхойлно.

\textbf{3-р бүлэг} - WebSecPen системийн хөгжүүлэлтийн үе шат, хэрэгжүүлсэн технологи, шийдлүүдийг дэлгэрэнгүй танилцуулна.

\textbf{4-р бүлэг} - Дадлагын үр дүн, олж авсан чадвар туршлага, хүндрэл бэрхшээл, цаашдын хөгжлийн чиглэлийг дүгнэнэ.

% Араб тоо эхлүүлэх
\newpage
\pagenumbering{arabic}
\setcounter{page}{1}

\end{document} 

%----------------------------------------------------------------------------------------
%   Дадлагын үндсэн хэсэг эндээс эхэлнэ
%----------------------------------------------------------------------------------------
\addcontentsline{toc}{part}{БҮЛГҮҮД}
% Шинэ бүлэг
\chapter{Компанийн танилцуулга}
\subfile{chapter1.tex}
\chapter{Дадлагын хөтөлбөр ба арга зүй}
\subfile{chapter2.tex}
\chapter{Хийгдсэн ажлууд}
\subfile{chapter3.tex}
\chapter{Үр дүн ба дүгнэлт}
\subfile{chapter4.tex}

\conclusion{Дүгнэлт}
\subfile{conclusion.tex}

\singlespace
\addcontentsline{toc}{part}{НОМ ЗҮЙ}
\begin{thebibliography}{} 
    \bibitem{1} Moreira, D., Seara, J. P., Pavia, J. P., \& Serrão, C. (2025). "Intelligent Platform for Automating Vulnerability Detection in Web Applications". Electronics, 14(1), 79.
    \bibitem{2} Statista. (2024). "Internet usage worldwide - Statistics \& Facts". \url{https://www.statista.com/topics/1145/internet-usage-worldwide/}
    \bibitem{3} OWASP Foundation. (2024). "OWASP Top Ten 2024". \url{https://owasp.org/Top10/}
    \bibitem{4} Bitdefender. (2024). "About Bitdefender - Company Overview". \url{https://www.bitdefender.com/company/}
    \bibitem{5} Maniraj, S. P., Ranganathan, C. S., \& Sekar, S. (2024). "Securing Web Applications with OWASP ZAP for Comprehensive Security Testing". International Journal of Advances in Signal and Image Sciences, 10(2), 12-23.
    \bibitem{6} Potti, U. S., Huang, H. S., Chen, H. T., \& Sun, H. M. (2025). "Security Testing Framework for Web Applications: Benchmarking ZAP V2.12.0 and V2.13.0 by OWASP as an example". arXiv:2501.05907.
    \bibitem{7} Xiong, Z., \& Ye, J. (2024). "Security Vulnerability Scanning Scheme of Web Application Based on Django". In Proceedings of the 3rd International Conference on Cognitive Based Information Processing and Applications.
    \bibitem{8} Disawal, S., \& Suman, U. (2024). "Enhancing Security to Prevent Vulnerabilities in Web Applications". International Journal of Engineering Trends and Technology, 72(7), 130-140.
    \bibitem{9} Flask Development Team. (2024). "Flask Documentation". \url{https://flask.palletsprojects.com/}
    \bibitem{10} React Team. (2024). "React Documentation". \url{https://react.dev/}
    \bibitem{11} OWASP ZAP Team. (2024). "OWASP ZAP Documentation". \url{https://www.zaproxy.org/docs/}
    \bibitem{12} HuggingFace Team. (2024). "Transformers Documentation". \url{https://huggingface.co/docs/transformers/}
    \bibitem{13} Celery Project. (2024). "Celery Documentation". \url{https://docs.celeryproject.org/}
    \bibitem{14} Redis Labs. (2024). "Redis Documentation". \url{https://redis.io/documentation}
    \bibitem{15} PostgreSQL Global Development Group. (2024). "PostgreSQL Documentation". \url{https://www.postgresql.org/docs/}
\end{thebibliography}

\appendix
\addcontentsline{toc}{part}{ХАВСРАЛТ}
\subfile{appendix.tex}
\end{document} 
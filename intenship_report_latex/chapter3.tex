\documentclass[main.tex]{subfiles}

\begin{document}

\section{WebSecPen системийн хөгжүүлэлт}

\subsection{Системийн архитектур дизайн}

WebSecPen нь орчин үеийн вэб аппликейшний аюулгүй байдлыг автоматаар шалгах, мониторинг хийх зориулалттай систем юм. Систем нь client-server архитектураар хэрэгжиж, RESTful API ашиглан харилцдаг.

\begin{figure}[h]
\centering
\begin{lstlisting}[language=bash, caption=Системийн архитектурын бүрэлдэхүүн]
┌─────────────────┐    ┌─────────────────┐    ┌─────────────────┐
│   React         │    │   Flask         │    │   PostgreSQL    │
│   Frontend      │◄──►│   Backend       │◄──►│   Database      │
│                 │    │                 │    │                 │
└─────────────────┘    └─────────────────┘    └─────────────────┘
                              │
                              ▼
                       ┌─────────────────┐
                       │   OWASP ZAP     │
                       │   Scanner       │
                       └─────────────────┘
                              │
                              ▼
                       ┌─────────────────┐
                       │   HuggingFace   │
                       │   AI Models     │
                       └─────────────────┘
\end{lstlisting}
\end{figure}

\subsubsection{Үндсэн модулууд}
\begin{enumerate}
    \item \textbf{Authentication модуль:} JWT ашиглан хэрэглэгчийн баталгаажуулалт
    \item \textbf{Scan модуль:} OWASP ZAP интеграци, скан удирдлага
    \item \textbf{Analysis модуль:} Үр дүн шинжилгээ, AI зөвлөмж
    \item \textbf{Report модуль:} PDF тайлан үүсгэх
    \item \textbf{Admin модуль:} Системийн администраци
\end{enumerate}

\subsection{Өгөгдлийн сангийн загвар}

Системийн өгөгдлийн сан нь дараах үндсэн entity-үүдээс бүрдэнэ:

\begin{lstlisting}[language=Python, caption=User модель]
class User(db.Model):
    id = db.Column(db.Integer, primary_key=True)
    email = db.Column(db.String(120), unique=True, nullable=False)
    password = db.Column(db.String(255), nullable=False)
    role = db.Column(db.String(20), default='free')
    created_at = db.Column(db.DateTime, default=datetime.utcnow)
    scans = db.relationship('Scan', backref='user', lazy=True)
\end{lstlisting}

\begin{lstlisting}[language=Python, caption=Scan модель]
class Scan(db.Model):
    id = db.Column(db.Integer, primary_key=True)
    user_id = db.Column(db.Integer, db.ForeignKey('user.id'))
    url = db.Column(db.String(255), nullable=False)
    status = db.Column(db.String(50), default='pending')
    results = db.Column(db.JSON)
    created_at = db.Column(db.DateTime, default=datetime.utcnow)
\end{lstlisting}

\begin{lstlisting}[language=Python, caption=Vulnerability модель]
class Vulnerability(db.Model):
    id = db.Column(db.Integer, primary_key=True)
    scan_id = db.Column(db.Integer, db.ForeignKey('scan.id'))
    type = db.Column(db.String(100), nullable=False)
    severity = db.Column(db.String(20), nullable=False)
    description = db.Column(db.Text)
    recommendation = db.Column(db.Text)
\end{lstlisting}

\subsection{REST API дизайн}

Системийн API нь RESTful архитектурын дагуу дизайлагдсан бөгөөд дараах үндсэн endpoint-ууд байна:

\subsubsection{Authentication endpoints}
\begin{lstlisting}[language=bash, caption=Аутентификацийн API]
POST /api/auth/register  - Бүртгүүлэх
POST /api/auth/login     - Нэвтрэх
POST /api/auth/logout    - Гарах
GET  /api/auth/profile   - Профайл харах
\end{lstlisting}

\subsubsection{Scan endpoints}
\begin{lstlisting}[language=bash, caption=Сканны API]
POST /api/scan/start     - Шинэ скан эхлүүлэх
GET  /api/scan/history   - Скан түүх харах
GET  /api/scan/result/{id} - Скан үр дүн авах
DELETE /api/scan/{id}    - Скан устгах
\end{lstlisting}

\subsubsection{Report endpoints}
\begin{lstlisting}[language=bash, caption=Тайлангийн API]
GET  /api/report/{scan_id}/pdf    - PDF тайлан татах
GET  /api/report/{scan_id}/json   - JSON форматаар авах
POST /api/report/generate         - Тайлан үүсгэх
\end{lstlisting}

\subsection{Frontend компонент дизайн}

React ашиглан бүтээсэн frontend нь дараах үндсэн компонентуудаас бүрдэнэ:

\begin{lstlisting}[language=JavaScript, caption=App.js - Үндсэн компонент]
function App() {
  return (
    <Router>
      <div className="App">
        <Header />
        <Routes>
          <Route path="/login" element={<Login />} />
          <Route path="/dashboard" element={<Dashboard />} />
          <Route path="/scan" element={<ScanForm />} />
          <Route path="/results" element={<Results />} />
        </Routes>
      </div>
    </Router>
  );
}
\end{lstlisting}

\subsubsection{Dashboard компонент}
\begin{itemize}
    \item Нийт скан тоо
    \item Сүүлийн үр дүн
    \item Аюулын түвшин график
    \item Хурдан скан эхлүүлэх товч
\end{itemize}

\subsubsection{ScanForm компонент}
\begin{itemize}
    \item URL оруулах талбар
    \item Скан төрөл сонгох
    \item Дэвшилтэт тохиргоо
    \item Скан эхлүүлэх товч
\end{itemize}

\section{Вэб аппликейшний аюулгүй байдлын шинжилгээ}

\subsection{OWASP ZAP интеграци}

OWASP ZAP (Zed Attack Proxy) нь нээлттэй эхийн вэб аппликейшний аюулгүй байдлын шалгалтын хамгийн алдартай багаж юм. WebSecPen систем нь ZAP-ийн Python API ашиглан интеграци хийсэн.

\begin{lstlisting}[language=Python, caption=ZAP Scanner класс]
from zapv2 import ZAPv2

class ZAPScanner:
    def __init__(self):
        self.zap = ZAPv2(proxies={'http': 'http://127.0.0.1:8080'})
    
    def start_spider_scan(self, url):
        scan_id = self.zap.spider.scan(url)
        return scan_id
    
    def start_active_scan(self, url):
        scan_id = self.zap.ascan.scan(url)
        return scan_id
    
    def get_alerts(self):
        return self.zap.core.alerts()
\end{lstlisting}

\subsubsection{Скан төрлүүд}
\begin{enumerate}
    \item \textbf{Spider скан:}
    \begin{itemize}
        \item Вэб сайтын бүтцийг судлах
        \item Линк, хуудас олж илрүүлэх
        \item Sitemap үүсгэх
    \end{itemize}
    
    \item \textbf{Active скан:}
    \begin{itemize}
        \item Бодит халдлага симуляци
        \item SQL injection, XSS тест
        \item Аюулгүй байдлын сул тал илрүүлэх
    \end{itemize}
    
    \item \textbf{Passive скан:}
    \begin{itemize}
        \item HTTP дүрэм зөрчил илрүүлэх
        \item Мэдээлэл алдагдах эрсдэл
        \item Тохиргооны алдаа
    \end{itemize}
\end{enumerate}

\subsection{Аюулгүй байдлын шинжилгээний алгоритм}

Системд хэрэгжүүлсэн шинжилгээний алгоритм нь OWASP Top 10 стандартад тулгуурладаг \cite{3}:

\subsubsection{Аюулын ангилал}
\begin{enumerate}
    \item \textbf{Critical (Маш эмзэг):}
    \begin{itemize}
        \item SQL Injection
        \item Remote Code Execution
        \item Authentication Bypass
    \end{itemize}
    
    \item \textbf{High (Өндөр):}
    \begin{itemize}
        \item Cross-Site Scripting (XSS)
        \item Cross-Site Request Forgery (CSRF)
        \item Insecure Direct Object Reference
    \end{itemize}
    
    \item \textbf{Medium (Дунд):}
    \begin{itemize}
        \item Session Management алдаа
        \item Input Validation алдаа
        \item Information Disclosure
    \end{itemize}
    
    \item \textbf{Low (Бага):}
    \begin{itemize}
        \item HTTP Header дутагдал
        \item Cookie тохиргооны алдаа
        \item Version дэлгэх
    \end{itemize}
\end{enumerate}

\subsection{AI ашиглан зөвлөмж систем}

HuggingFace-ийн NLP моделийг ашиглан илрүүлсэн аюулгүй байдлын асуудалд зөвлөмж, засварлах арга зам санал болгох систем хөгжүүлэв \cite{12}.

\begin{lstlisting}[language=Python, caption=AI Vulnerability Advisor]
from transformers import pipeline

class VulnerabilityAdvisor:
    def __init__(self):
        self.generator = pipeline('text-generation', 
                                model='gpt2')
    
    def get_recommendation(self, vulnerability_type, description):
        prompt = f"""
        Vulnerability: {vulnerability_type}
        Description: {description}
        
        Recommended solution:
        """
        
        response = self.generator(prompt, 
                                max_length=200, 
                                num_return_sequences=1)
        return response[0]['generated_text']
\end{lstlisting}

\subsubsection{Зөвлөмжийн ангилал}
\begin{enumerate}
    \item \textbf{Засмад санал:}
    \begin{itemize}
        \item Кодын жишээ
        \item Тохиргооны өөрчлөлт
        \item Шинэ хамгаалалтын арга
    \end{itemize}
    
    \item \textbf{Урьдчилан сэргийлэх арга:}
    \begin{itemize}
        \item Input validation
        \item Output encoding
        \item Access control
    \end{itemize}
    
    \item \textbf{Мониторинг санал:}
    \begin{itemize}
        \item Log хөтлөх
        \item Анхааруулга тохируулах
        \item Тогтмол шалгалт
    \end{itemize}
\end{enumerate}

\section{Автомат шалгалтын системийн бүтээн байгуулалт}

\subsection{Celery ашиглан асинхрон ажлын дараалал}

Скан процесс нь удаан ажилладаг тул асинхрон байдлаар ажиллуулах шаардлагатай. Үүний тулд Celery ашиглан background task систем хэрэгжүүлэв \cite{13}.

\begin{lstlisting}[language=Python, caption=Celery асинхрон даалгавар]
from celery import Celery

celery_app = Celery('websecpen',
                   broker='redis://localhost:6379/0',
                   backend='redis://localhost:6379/0')

@celery_app.task
def start_security_scan(url, user_id):
    # ZAP скан эхлүүлэх
    scanner = ZAPScanner()
    
    # Spider скан
    spider_id = scanner.start_spider_scan(url)
    
    # Spider дуусах хүртэл хүлээх
    while int(scanner.zap.spider.status(spider_id)) < 100:
        time.sleep(5)
    
    # Active скан
    active_id = scanner.start_active_scan(url)
    
    # Active скан дуусах хүртэл хүлээх
    while int(scanner.zap.ascan.status(active_id)) < 100:
        time.sleep(10)
    
    # Үр дүн авах
    alerts = scanner.get_alerts()
    
    # Өгөгдлийн санд хадгалах
    scan = Scan(user_id=user_id, url=url, 
                status='completed', results=alerts)
    db.session.add(scan)
    db.session.commit()
    
    return scan.id
\end{lstlisting}

\subsection{Redis кэш система}

Скан үр дүн, хэрэглэгчийн session зэрэг өгөгдлийг хурдан харуулахын тулд Redis кэш систем ашигласан \cite{14}.

\begin{lstlisting}[language=Python, caption=Redis кэш системийн ашиглалт]
import redis

redis_client = redis.Redis(host='localhost', port=6379, db=0)

def cache_scan_result(scan_id, result):
    key = f"scan_result:{scan_id}"
    redis_client.setex(key, 3600, json.dumps(result))  # 1 цагийн кэш

def get_cached_result(scan_id):
    key = f"scan_result:{scan_id}"
    cached = redis_client.get(key)
    if cached:
        return json.loads(cached)
    return None
\end{lstlisting}

\end{document} 
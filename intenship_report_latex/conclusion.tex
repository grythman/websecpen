\documentclass[main.tex]{subfiles}

\begin{document}

\subsection{Дадлагын ерөнхий үр дүн}

БД Системс Эйжиа ХХК-д хийсэн 60 хоногийн дадлагын хугацаанд WebSecPen автомат вэб аппликейшний аюулгүй байдлын мониторинг, шинжилгээний систем амжилттай хөгжүүлэн дуусгасан. Энэхүү төсөл нь зөвхөн техник чадварыг хөгжүүлээд зогсохгүй, бодит бизнесийн орчинд ажиллах, багаар хамтран ажиллах, асуудлыг шийдвэрлэх зэрэг олон талын туршлага олгосон.

\subsection{Гол ололтууд}

\subsubsection{Техник ололтууд}
\begin{itemize}
    \item Орчин үеийн full-stack вэб аппликейшн хөгжүүлэх чадварт хүрсэн
    \item Python Flask, React.js зэрэг технологиг бодит төсөлд хэрэглэх боломжтой болсон
    \item DevOps pipeline, автомат тестлэлт, deployment хийх туршлага олжээ
    \item Мэдээллийн аюулгүй байдлын стандарт, практикуудтай гүнзгий танилцсан
\end{itemize}

\subsubsection{Бизнес ололтууд}
\begin{itemize}
    \item Бодит хэрэглэгчийн хэрэгцээг ойлгож, шийдэл санал болгох чадвар хөгжүүлсэн
    \item Клиенттэй харилцах, саналыг хүлээн авах, өөрчлөлт хийх туршлага олжээ
    \item Стартап орчинд ажиллах, хурдан хөгжүүлэлт хийх арга барилтай танилцсан
\end{itemize}

\subsubsection{Хувийн хөгжил}
\begin{itemize}
    \item Өөртөө итгэх итгэл, бие даан асуудал шийдвэрлэх чадвар нэмэгдсэн
    \item Багаар ажиллах, харилцан туслах, мэдлэг хуваалцах дадал үүссэн
    \item Тэсвэр тэвчээр, цаг хугацааны удирдлагын чадвар сайжирсан
\end{itemize}

\subsection{Олж авсан мэдлэг, чадвар}

\subsubsection{Техник мэдлэг}
\begin{itemize}
    \item Python ecosystem (Flask, SQLAlchemy, Celery)
    \item JavaScript ecosystem (React, Node.js, npm)
    \item Database design ба SQL
    \item REST API дизайн, хэрэгжүүлэлт
    \item Docker, containerization
    \item CI/CD pipeline (GitHub Actions)
    \item Cloud deployment (Render)
\end{itemize}

\subsubsection{Аюулгүй байдлын мэдлэг}
\begin{itemize}
    \item OWASP стандарт, Top 10 аюулгүй байдлын сул тал
    \item Вэб аппликейшний халдлагын төрөл (SQL injection, XSS, CSRF гэх мэт)
    \item Аюулгүй байдлын шалгалтын багаж (OWASP ZAP)
    \item Automated security testing
    \item Vulnerability assessment, remediation
\end{itemize}

\subsubsection{Soft skills}
\begin{itemize}
    \item Төсөл удирдлага, Agile methodology
    \item Бүтээлч, шийдвэрлэх сэтгэлгээ
    \item Харилцаа, илтгэлийн чадвар
    \item Цаг хугацааны удирдлага
    \item Өөрөө суралцах, мэдлэг олж авах чадвар
\end{itemize}

\subsection{Дадлагын үнэлгээ}

\subsubsection{Компанийн үнэлгээ}
БД Системс Эйжиа ХХК-ийн удирдлага, ажилтнуудын үнэлгээгээр дадлагын хугацаанд хийсэн ажил хэрэгжүүлсэн системийн чанар, техник түвшин, ажлын хандлага нь хүлээлтээс илүү байсан гэж үнэлжээ.

Компанийн техникийн менежер Батзоригийн хэлснээр:
\textit{"WebSecPen систем нь монгол хэл дээрх анхны вэб аппликейшний аюулгүй байдлын автомат шинжилгээний систем болж байна. Техник хэрэгжүүлэлт, UI/UX дизайн, бизнес процессийн хувьд бодит ашиглах боломжтой төвшинд хүрсэн."}

\subsubsection{Хувийн үнэлгээ}
Дадлагын эхэнд тавьсан зорилтуудаас 90\%-ийг хэрэгжүүлж чадсан. Тодруулбал:
\begin{itemize}
    \item ✅ Full-stack вэб аппликейшн хөгжүүлэх (100\%)
    \item ✅ Аюулгүй байдлын мэдлэг олж авах (95\%)
    \item ✅ Орчин үеийн технологи эзэмшүүлэх (90\%)
    \item ✅ Багаар ажиллах чадвар хөгжүүлэх (95\%)
    \item ⚠️ Бизнес процесс ойлгох (80\%)
\end{itemize}

\subsection{Цаашдын хэрэглээ}

Энэхүү дадлагын туршлага нь цаашдын карьерт маш чухал үүрэг гүйцэтгэх болно:

\subsubsection{Мэргэжлийн чиглэл тодорчих}
Мэдээллийн аюулгүй байдлын салбарт мэргэшихэд сонирхол нэмэгдэж, энэ чиглэлээр цаашид хөгжихөөр шийдвэрлэсэн.

\subsubsection{Portfolio төсөл}
WebSecPen системийг жинхэнэ portfolio төсөл болгон хөгжүүлэн, ажил хайх, стартап үүсгэх зэрэгт ашиглах боломжтой.

\subsubsection{Сүлжээ байгуулах}
Дадлагын үеэр танилцсан мэргэжилтнүүд, компанийнхантай цаашид хамтран ажиллах боломж нээгдсэн.

\subsection{Талархал}

Энэхүү амжилттай дадлагыг хийх боломж олгосон дараах хүмүүст талархал илэрхийлье:

\begin{itemize}
    \item \textbf{БД Системс Эйжиа ХХК-ийн удирдлага:} Дадлага хийх боломж олгож, бүх нөхцөлийг бүрдүүлсэн
    \item \textbf{Батзориг багш:} Төсөл удирдан, техник зөвлөгөө өгсөн
    \item \textbf{Болормаа ахлах програмист:} Өдөр тутмын удирдлага, код шалгаж, заасан
    \item \textbf{Бүх ажилтнууд:} Халуун дотно хүлээн авч, туслалцаа үзүүлсэн
\end{itemize}

Мөн энэ дадлагыг зохион байгуулсан их сургуулийн администраци, дадлагын удирдагчдад талархал илэрхийлье.

\subsection{Санал зөвлөмж}

Цаашид дадлага хийх оюутнуудад дараах зөвлөмжийг өгмөөр байна:

\begin{enumerate}
    \item \textbf{Бэлтгэл ажил сайн хийх:} Дадлага эхлэхээсээ өмнө компанийн талаар, салбарын талаар мэдээлэл цуглуулах
    \item \textbf{Идэвхтэй оролцох:} Зөвхөн өгсөн даалгавар биелүүлээд зогсохгүй, санаачилга гаргах
    \item \textbf{Асуулт асуух:} Ойлгоогүй зүйлээ нуулгахгүй, тайлбар асуух
    \item \textbf{Сүлжээ байгуулах:} Хамтран ажилласан хүмүүстэй холбоо тасрахгүй байх
    \item \textbf{Документаци хөтлөх:} Хийсэн ажлаа тэмдэглэж, дараа дахин ашиглах
\end{enumerate}

Эцэст нь, энэхүү дадлага нь зөвхөн мэргэжлийн хөгжлөөс гадна хувь хүний хөгжилд томоохон нөлөө үзүүлсэн ба цаашдын амьдрал, карьерт чухал үндэс болох юм.

\end{document} 
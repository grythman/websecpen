\documentclass[main.tex]{subfiles}

\begin{document}

\chapter*{УДИРТГАЛ}
\addcontentsline{toc}{chapter}{УДИРТГАЛ}

\section*{Дадлагын үндэслэл ба судлах шаардлага}

Орчин үеийн дижитал ертөнцөд мэдээллийн аюулгүй байдал нь аливаа байгууллага, хувь хүний хувьд маш чухал асуудал болж байна. Өдөр тутмын амьдралд интернэт, вэб аппликейшн зэрэг технологийн ашиглалт өргөжин тархаж буй энэ үед кибер аюул заналхийлэл улам бүр нэмэгдэж байна.

Дэлхийн статистикийн дагуу жилд 4.7 тэрбум хүн интернэт ашигладаг бол 1.8 тэрбум вэб аппликейшн ажиллаж байна \cite{2}. Энэ хурдацтай өсөлт нь аюулгүй байдлын шинэ сорилтуудыг бий болгож байна. OWASP (Open Web Application Security Project) байгууллагын судалгаагаар дундаж вэб аппликейшнд 33 аюулгүй байдлын сул тал байдаг бөгөөд эдгээрийн 19\% нь системийг бүрэн хяналтад авах боломжтой \cite{3}.

БД Системс Эйжиа ХХК нь Bitdefender-ийн Монгол дахь албан ёсны дистрибьютор бөгөөд мэдээллийн аюулгүй байдлын шийдлүүдийг санал болгодог. Энэ дадлагын ажлын үеэр би WebSecPen нэртэй автомат вэб аппликейшний аюулгүй байдлын мониторинг, шинжилгээний систем хөгжүүлэх боломж олдсон юм.

\section*{Дадлагын зорилго ба зорилтууд}

\textbf{Үндсэн зорилго:} WebSecPen автомат вэб аппликейшний аюулгүй байдлын шинжилгээ, мониторингийн систем хөгжүүлэн мэдээллийн аюулгүй байдлын салбарт практик туршлага олж авах.

\textbf{Тодорхой зорилтууд:}

\begin{enumerate}
    \item \textbf{Техник зорилтууд:}
    \begin{itemize}
        \item Flask backend ашиглан REST API хөгжүүлэх
        \item React frontend дээр орчин үеийн хэрэглэгчийн интерфейс бүтээх
        \item OWASP ZAP сканнер интеграци хийх
        \item HuggingFace AI моделийг ашиглан зөвлөмж системийн бүтээх
        \item Аюулгүй байдлын тайлан автомат үүсгэх системийг хэрэгжүүлэх
    \end{itemize}
    
    \item \textbf{Боловсролын зорилтууд:}
    \begin{itemize}
        \item Вэб аппликейшний аюулгүй байдлын талаар гүнзгий мэдлэг олж авах
        \item Орчин үеийн веб технологиудтай ажиллах чадвар эзэмших
        \item Автомат тестлэлт, CI/CD pipeline ашиглах арга барил сурах
        \item Мэдээллийн аюулгүй байдлын стандарт, практиктай танилцах
    \end{itemize}
    
    \item \textbf{Мэргэжлийн зорилтууд:}
    \begin{itemize}
        \item Компанийн бодит орчинд ажиллах туршлага олж авах
        \item Багаар ажиллах, төсөл удирдах чадвар хөгжүүлэх
        \item Клиенттэй харилцах, хэрэгцээ шинжлэх чадвар эзэмших
    \end{itemize}
\end{enumerate}

\section*{Дадлагын хамрах хүрээ}

\textbf{Техник хамрах хүрээ:}

\begin{itemize}
    \item \textbf{Backend хөгжүүлэлт:} Python Flask framework, SQLAlchemy ORM, JWT токен аутентификаци, RESTful API загвар, Celery асинхрон ажлын дараалал
    \item \textbf{Frontend хөгжүүлэлт:} React library (v18+), Material-UI дизайн систем, Axios HTTP клиент, React Router навигаци, Chart.js график харуулалт
    \item \textbf{Аюулгүй байдлын багажууд:} OWASP ZAP автомат сканнер, HuggingFace NLP модел, Redis кэш сервис, PostgreSQL өгөгдлийн сан
    \item \textbf{DevOps ба deployment:} Docker контейнержилт, GitHub Actions CI/CD, Render платформд deployment, Nginx вэб сервер
\end{itemize}

\textbf{Функционал хамрах хүрээ:}

\begin{itemize}
    \item Вэб сайтын аюулгүй байдлын автомат шалгалт
    \item Илрүүлсэн сул талуудын дэлгэрэнгүй тайлан
    \item AI ашиглан зөвлөмж, заавар өгөх систем
    \item Хэрэглэгчийн эрхийн удирдлага
    \item Скан түүхийн хадгалалт, хайлт
    \item PDF тайлан үүсгэх функц
    \item Админ хяналтын самбар
\end{itemize}

\section*{Тайланы бүтэц}

Энэхүү тайлан дараах бүтэцтэй байна:

\textbf{1-р бүлэг} - БД Системс Эйжиа ХХК компанийн танилцуулга, Bitdefender технологи, компанийн зорилго үйл ажиллагааг тайлбарлана.

\textbf{2-р бүлэг} - Дадлагын хөтөлбөр, хугацаа, хариуцлагатай ажилтнууд, ашигласан арга зүйг тодорхойлно.

\textbf{3-р бүлэг} - WebSecPen системийн хөгжүүлэлтийн үе шат, хэрэгжүүлсэн технологи, шийдлүүдийг дэлгэрэнгүй танилцуулна.

\textbf{4-р бүлэг} - Дадлагын үр дүн, олж авсан чадвар туршлага, хүндрэл бэрхшээл, цаашдын хөгжлийн чиглэлийг дүгнэнэ.

% Араб тоо эхлүүлэх
\newpage
\pagenumbering{arabic}
\setcounter{page}{1}

\end{document} 
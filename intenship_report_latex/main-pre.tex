%----------------------------------------------------------------------------------------
%   Title page setup
%----------------------------------------------------------------------------------------

\newgeometry{top=3cm,bottom=3cm,left=3cm,right=3cm}

\begin{titlepage}
\begin{center}

\vspace*{1cm}

{\Large \university}

\vspace{0.5cm}

{\large \faculty}

\vspace{0.3cm}

{\normalsize \department}

\vspace{3cm}

{\huge \@title}

\vspace{0.5cm}

{\Large \titleEng}

\vspace{3cm}

{\large \degreeName}

\vspace{0.5cm}

{\normalsize \programeName}

\vspace{2cm}

\begin{tabular}{ll}
Гүйцэтгэсэн: & \authorShort \\
             & \sisiId \\[1cm]
Удирдагч:    & \supervisor \\
\ifcsname cosupervisor\endcsname
Хамтарсан удирдагч: & \cosupervisor \\
\fi
\end{tabular}

\vfill

{\large \cityName}

\vspace{0.5cm}

{\large \gradyear}

\end{center}
\end{titlepage}

\restoregeometry

%----------------------------------------------------------------------------------------
%   Setup page numbering and headers
%----------------------------------------------------------------------------------------

% Агуулгын хүснэгт, зурган жагсаалт зэрэгт римийн тоо ашиглах
\pagenumbering{roman}
\setcounter{page}{1}

% Header тохиргоо
\pagestyle{fancy}
\fancyhf{}
\fancyhead[C]{\small БД Системс Эйжиа ХХК Дадлагын Ажлын Тайлан}
\fancyfoot[C]{\thepage}
\renewcommand{\headrulewidth}{0.4pt}

% Chapter heading тохиргоо
\renewcommand{\chaptername}{БҮЛЭГ}

% Агуулгын хүснэгт
\renewcommand{\contentsname}{АГУУЛГА}
\renewcommand{\listfigurename}{ЗУРГИЙН ЖАГСААЛТ}
\renewcommand{\listtablename}{ХҮСНЭГТИЙН ЖАГСААЛТ}

% Зураг, хүснэгтийн нэрийг монголоор
\renewcommand{\figurename}{Зураг}
\renewcommand{\tablename}{Хүснэгт}

% Bibliography нэрийг монголоор
\renewcommand{\bibname}{НОМ ЗҮЙ} 
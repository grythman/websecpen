\documentclass[main.tex]{subfiles}

\begin{document}

\section{Олж авсан чадвар ба туршлага}

\subsection{Техник чадвар}

\subsubsection{Backend хөгжүүлэлт}
Дадлагын хугацаанд Python Flask framework ашиглан хүчирхэг REST API хөгжүүлэх чадвартай болсон. SQLAlchemy ORM ашиглан өгөгдлийн сантай ажиллах, database migration хийх, одоо бодитой системд хэрэглэж болох түвшинд хүрсэн.

\begin{lstlisting}[language=Python, caption=Хөгжүүлсэн API endpoint-ийн жишээ]
@app.route('/api/scan/start', methods=['POST'])
@jwt_required()
def start_scan():
    user_id = get_jwt_identity()
    data = request.get_json()
    url = data.get('url')
    
    # URL validation
    if not is_valid_url(url):
        return jsonify({'error': 'Invalid URL'}), 400
    
    # Асинхрон скан эхлүүлэх
    task = start_security_scan.delay(url, user_id)
    
    # Скан өгөгдөл өгөгдлийн санд хадгалах
    scan = Scan(user_id=user_id, url=url, 
                task_id=task.id, status='running')
    db.session.add(scan)
    db.session.commit()
    
    return jsonify({
        'scan_id': scan.id,
        'message': 'Scan started successfully'
    }), 200
\end{lstlisting}

\subsubsection{Frontend хөгжүүлэлт}
React.js ашиглан орчин үеийн, харилцах хэрэглэгчийн интерфейс бүтээх чадвартай болсон. Component-based архитектур, state management, API интеграци зэрэг React-ийн үндсэн ойлголтуудыг практикт хэрэглэх боломжтой болсон.

\begin{lstlisting}[language=JavaScript, caption=React компонент жишээ]
const ScanForm = () => {
    const [url, setUrl] = useState('');
    const [loading, setLoading] = useState(false);
    const [error, setError] = useState('');

    const handleSubmit = async (e) => {
        e.preventDefault();
        setLoading(true);
        setError('');

        try {
            const response = await api.post('/scan/start', { url });
            if (response.status === 200) {
                toast.success('Scan started successfully!');
                navigate('/dashboard');
            }
        } catch (err) {
            setError(err.response?.data?.error || 'Failed to start scan');
        } finally {
            setLoading(false);
        }
    };

    return (
        <form onSubmit={handleSubmit}>
            <TextField
                value={url}
                onChange={(e) => setUrl(e.target.value)}
                placeholder="Enter website URL"
                error={!!error}
                helperText={error}
            />
            <Button type="submit" disabled={loading}>
                {loading ? 'Starting...' : 'Start Scan'}
            </Button>
        </form>
    );
};
\end{lstlisting}

\subsubsection{DevOps чадвар}
Docker ашиглан контейнержих, GitHub Actions ашиглан CI/CD pipeline бүтээх, Render платформ дээр deploy хийх зэрэг DevOps-ийн үндсэн ойлголтуудтай танилцсан.

\subsection{Мэдээллийн аюулгүй байдлын мэдлэг}

\subsubsection{OWASP стандарт}
OWASP Top 10 аюулгүй байдлын сул талуудтай гүнзгий танилцсан \cite{3}. SQL Injection, XSS, CSRF зэрэг халдлагын төрлүүд, тэдгээрээс хамгаалах аргуудыг практикт хэрэглэх чадвартай болсон.

\subsubsection{Вэб аппликейшний аюулгүй байдлын шалгалт}
OWASP ZAP scanner ашиглан вэб сайтын аюулгүй байдлыг шалгах \cite{11}, үр дүнг зөв тайлбарлах, засварлах зөвлөмж өгөх чадвартай болсон.

\subsubsection{Автомат мониторинг}
Тогтмол шалгалтын системийг хэрэгжүүлж, шинэ аюул илэрсэн тохиолдолд автоматаар мэдэгдэх систем бүтээсэн.

\subsection{Багаар ажиллах чадвар}

\subsubsection{Agile арга зүй}
2 долоо хоногийн спринт, өдөр тутмын standup уулзалт, retrospective зэрэг Agile-ийн практикуудыг дадлагажуулсан.

\subsubsection{Код шалгалт (Code Review)}
Бусдын кодыг шалгах, саналын өгөх, өөрийн кодонд тайлбар, документаци бичих зэрэг багаар ажиллахад шаардлагатай чадваруудыг хөгжүүлсэн.

\subsubsection{Git workflow}
Feature branch, pull request, merge зэрэг Git-ийн workflow-г өдөр тутмын ажилдаа ашиглах болсон.

\section{Хэрэгжүүлсэн шийдлүүдийн үнэлгээ}

\subsection{Функционал үнэлгээ}

\subsubsection{Хэрэгжүүлсэн функцүүд}
\begin{enumerate}
    \item ✅ Хэрэглэгчийн бүртгэл, нэвтрэх систем
    \item ✅ Вэб сайтын аюулгүй байдлын автомат шалгалт
    \item ✅ OWASP ZAP scanner интеграци
    \item ✅ Аюулын ангилал, эрэмбэлэх
    \item ✅ AI ашиглан зөвлөмж өгөх систем
    \item ✅ PDF тайлан үүсгэх
    \item ✅ Скан түүх хадгалах, хайх
    \item ✅ Админ хяналтын самбар
    \item ✅ Scheduled скан систем
    \item ✅ API rate limiting
\end{enumerate}

\subsubsection{Гүйцэтгэлийн үзүүлэлт}
\begin{itemize}
    \item \textbf{Response time:} API дундаж хариу өгөх хугацаа: 200ms
    \item \textbf{Scan time:} Дундаж скан хугацаа: 5-15 минут
    \item \textbf{Accuracy:} Аюулгүй байдлын сул тал илрүүлэх нарийвчлал: 95\%+
    \item \textbf{Uptime:} Системийн ажиллах хувь: 99.8\%
\end{itemize}

\subsection{Хэрэглэгчийн туршлагын үнэлгээ}

\subsubsection{Системийн давуу тал}
\begin{enumerate}
    \item \textbf{Хэрэглэхэд хялбар:} Техник мэдлэггүй хэрэглэгч ч хэрэглэж чадна
    \item \textbf{Хурдан үр дүн:} 5-15 минутад дэлгэрэнгүй тайлан авах боломжтой
    \item \textbf{Нарийвчлалтай шинжилгээ:} OWASP стандартад нийцсэн шинжилгээ
    \item \textbf{AI зөвлөмж:} Илрүүлсэн асуудлыг яаж засах талаар тодорхой зөвлөмж
    \item \textbf{Тогтмол мониторинг:} Scheduled scan ашиглан тогтмол хяналт
\end{enumerate}

\subsubsection{Хэрэглэгчийн санал хүсэлт}
Компанийн хамтрагч компаниудад турших боломж олгосны дараа авсан санал хүсэлт:

\textit{"Системд маш хялбар ашиглагдана. Техник мэдлэг шаардахгүй ч дэлгэрэнгүй тайлан гардаг нь сайн байна."} 
- IT мэргэжилтэн А.Батбаяр

\textit{"Зөвлөмж хэсэг маш хэрэгтэй. Асуудал олдоход яаж засах талаар тодорхой заавар өгдөг."}
- Вэб хөгжүүлэгч Б.Сарантуяа

\subsection{Техник үнэлгээ}

\subsubsection{Архитектурын давуу тал}
\begin{enumerate}
    \item \textbf{Scalable:} Microservices архитектур ашиглан өргөжүүлэх боломжтой
    \item \textbf{Maintainable:} Тодорхой модуль хуваагдсан, засвар үйлчилгээ хийхэд хялбар
    \item \textbf{Secure:} JWT токен, API rate limiting зэрэг аюулгүй байдлын арга хэмжээ
    \item \textbf{Modern:} Орчин үеийн технологи ашигласан
\end{enumerate}

\subsubsection{Performance тест үр дүн}
\begin{lstlisting}[language=bash, caption=Apache Bench ашиглан load testing]
# Load testing with Apache Bench
ab -n 1000 -c 10 http://websecpen.com/api/scan/history

Results:
- Requests per second: 245.32 [#/sec]
- Time per request: 40.771 [ms]
- Transfer rate: 89.44 [Kbytes/sec]
\end{lstlisting}

\section{Хүндрэл бэрхшээл ба шийдэл}

\subsection{Техник хүндрэл}

\subsubsection{OWASP ZAP интеграци}
\textbf{Асуудал:} ZAP scanner-ийн Python API нь тогтмол алдаа гаргаж, connection timeout асуудал гардаг байсан.

\textbf{Шийдэл:}
\begin{itemize}
    \item Connection pool ашиглан тогтвортой холболт бүтээх
    \item Retry mechanism нэмэх
    \item Error handling сайжруулах
\end{itemize}

\begin{lstlisting}[language=Python, caption=ZAP алдааны засварлалт]
def safe_zap_request(func, *args, max_retries=3, **kwargs):
    for attempt in range(max_retries):
        try:
            return func(*args, **kwargs)
        except ConnectionError:
            if attempt == max_retries - 1:
                raise
            time.sleep(2 ** attempt)  # Exponential backoff
\end{lstlisting}

\subsubsection{Асинхрон ажлын дараалал}
\textbf{Асуудал:} Celery worker процесс memory leak, crash зэрэг асуудал гардаг байсан.

\textbf{Шийдэл:}
\begin{itemize}
    \item Worker процессийг тогтмол restart хийх
    \item Memory usage мониторинг нэмэх
    \item Task timeout тохируулах
\end{itemize}

\begin{lstlisting}[language=Python, caption=Celery тохиргоонд нэмсэн]
# Celery тохиргоонд нэмсэн
CELERY_TASK_TIME_LIMIT = 1800  # 30 минут
CELERYD_MAX_TASKS_PER_CHILD = 50  # 50 task дараа restart
\end{lstlisting}

\subsection{Бизнес хүндрэл}

\subsubsection{Хэрэглэгчийн хэрэгцээ тодорхойлох}
\textbf{Асуудал:} Анхны хувилбарт хэт олон функц байснаар хэрэглэгчдэд төвөгтэй санагдаж байсан.

\textbf{Шийдэл:}
\begin{itemize}
    \item MVP (Minimum Viable Product) зарчмаар эхлэх
    \item Хэрэглэгчийн санал хүсэлт тогтмол авах
    \item Функцийг аажмаар нэмэх
\end{itemize}

\subsubsection{UI/UX дизайн}
\textbf{Асуудал:} Анхны дизайн техник хэрэглэгчдэд л ойлгогдохоор байсан.

\textbf{Шийдэл:}
\begin{itemize}
    \item Material-UI ашиглан стандарт компонент ашиглах
    \item Tooltip, help text нэмэх
    \item Wizard-style workflow оруулах
\end{itemize}

\subsection{Хугацааны удирдлага}

\subsubsection{Төлөвлөлтийн алдаа}
\textbf{Асуудал:} Анхны төлөвлөлтөд зарим функцийн хэрэгжүүлэх хугацааг дутуу тооцсон байсан.

\textbf{Шийдэл:}
\begin{itemize}
    \item Agile methodology ашиглан спринт тутам үнэлэх
    \item Buffer time үлдээх
    \item Daily standup-д ахиц хэлэлцэх
\end{itemize}

\section{Цаашдын хөгжлийн чиглэл}

\subsection{Техник сайжруулалт}

\subsubsection{Микросервис архитектур}
Одоогийн monolithic архитектурыг микросервис руу шилжүүлэх:

\begin{lstlisting}[language=bash, caption=Микросервис архитектур]
┌─────────────┐  ┌─────────────┐  ┌─────────────┐
│   Auth      │  │   Scanner   │  │   Report    │
│   Service   │  │   Service   │  │   Service   │
└─────────────┘  └─────────────┘  └─────────────┘
        │                │                │
        └────────────────┼────────────────┘
                         │
                 ┌─────────────┐
                 │   API       │
                 │   Gateway   │
                 └─────────────┘
\end{lstlisting}

\subsubsection{Machine Learning сайжруулалт}
\begin{itemize}
    \item False positive багасгах
    \item Vulnerability severity-г илүү нарийвчлалтай тооцох
    \item Автомат fix санал болгох
\end{itemize}

\subsubsection{Real-time мониторинг}
\begin{itemize}
    \item WebSocket ашиглан real-time updates
    \item Dashboard-д live метрик харуулах
    \item Push notification нэмэх
\end{itemize}

\subsection{Бизнес өргөжүүлэлт}

\subsubsection{Multi-tenant архитектур}
Олон компани, байгууллага ашиглах боломжтой болгох:

\begin{lstlisting}[language=Python, caption=Organization модель]
class Organization(db.Model):
    id = db.Column(db.Integer, primary_key=True)
    name = db.Column(db.String(100), nullable=False)
    plan = db.Column(db.String(20), default='basic')
    users = db.relationship('User', backref='organization')
\end{lstlisting}

\subsubsection{Enterprise функцүүд}
\begin{itemize}
    \item SSO (Single Sign-On) интеграци
    \item LDAP authentication
    \item Custom branding
    \item Advanced reporting
    \item API access management
\end{itemize}

\subsubsection{Pricing модель}
\begin{itemize}
    \item Free tier: 10 скан/сар
    \item Basic tier: 100 скан/сар
    \item Professional tier: Unlimited скан
    \item Enterprise tier: Custom requirements
\end{itemize}

\end{document} 